\chapter{Uvod}

Proučavanje strukture DNA u fokusu je znanstvenika od samog otrkića DNA.
Precizno čitanje genoma predstavlja velik izazov te su s vremenom razvijani sve brži i jeftiniji uređaji za što bolja očitanja.
Danas su uređaji i dovoljno brzi i jeftini, ali se javlja problem kratkih očitanja.
Bioinformatika, između ostalog, teži spojiti ta kratka očitanja u jedinstvenu sekvencu (genom).

Prilikom sastavljanja genoma, javlja se nekoliko problema.
Prvi je nepreciznost uređaja za sekvenciranje, što zahtjeva višestruka očitanja jednog genoma kako bi se, određenim metodama, mogao ustanoviti stvarni niz.
Također, danas se koriste metode temeljene na \emph{shotgun} sekvenciranju cijelog genoma pri čemu nemamo nikakvu informaciju o poretku pojedinih očitanja.
Treći otežavajući faktor uspješnog sastavljanja jedinstvene sekvence jest varijabilna duljina očitanja.
Uređaji druge generacije, koji treutno prevladavaju, rade očitanja veličine od nekoliko desetaka do par stotina nukleotida.
Treća generacija uređaja proizvodi dulja očitanja, od nekoliko tisuća nukleotida, ali imaju velik postotak pogreške -- od 15\% do čak 40\%.

Razvijeno je nekoliko algoritama koji se bave problemom sastavljanja genoma, a najkorišteniji su oni temeljeni na algoritmima nad grafovima.
Najčešće se koristi jedna od dviju osnovnih metoda: Preklapanje-Razmještaj-Konsenzus \emph{engl. Overlap-Layout-Consensus, OLC} metode temeljene na grafu preklapanja ili metode temeljene na \emph{de Bruijn} grafovima.
U ovom projektu, bavimo se konsenzus fazom OLC paradigme.

Ovaj dokument organiziran je na sljedeći način: u sljedećem poglavlju opisan je algoritam koji smo koristili za dobivanje konsenzusa.
Poglavlje \ref{sec:implementacija} kratko opisuje našu konkretnu implementaciju i karakteristike korištenih računala.
U poglavlju \ref{sec:eval} dana je usporedba vlastite implementacije s referentnim radom.
Poslijednje poglavlje sadrži zaključak cjelokupnog projekta.


