\chapter{Evaluacija}
\label{sec:eval}

Testni podatci dobiveni su na kolegiju\footnote{fer.unizg.hr/predmet/bio}.
Evaluacija je obavljena na genomima dviju bakterija: escherichia coli te fuge lambda.
Za svaku bakteriju dostupna su očitanja, genom nakon faze razmještanja te referentni genom s kojim uspoređujemo performanse svoga algoritma.
Cilj je bio generirati genom konsenzus fazom kako bi njegovo preklapanje s referentnim bilo što veće od preklapanja genoma nakon faze razmještanja s referentnim.

\section{DnaDiff iz MUMmer paketa}

\emph{MUMmer} \citep{kurtz2004versatile} je široko korišten paket otvorenog koda \engl{Open Source} za razne podzadatke bioinformatike.
Implementira razne module, a mi smo koristili \emph{dnadiff} -- modul dizajniran za evaluaciju slijedova dvaju vrlo sličnih genoma.
Pruža detaljne informacije o razlikama između dvaju genoma, ali generira i high-level datoteku s kvantificiranim razlikama.
Svoj algoritam vrednovali smo prema \emph{AvgIdentity} polju u navedenoj datoteci (out.report).
Broj predstavlja prosječno poklapanje referentnog genoma s našim.
Postoje podatci za 1-na-1 mapiranje (gdje su ponavljanja zanemarena) te M-na-M mapiranje.
Uglavnom su oba broja jednaka, pa stoga nije posvećeno previše pažnje na odabir određenoga.
U svim rezultatima u nastavku nalazi se \emph{AvgIdentity} polje iz 1-na-1 mapiranja.

\section{Usporedba vlastitog rješenja s referentnim radom}

Tablica \ref{tbl:usporedba} prikazuje naše rezultate u usporedbi s referentnim radom. Također, prikazana je mjera preklapanja sekvence iz faze razmještanja s konačnom, referentnom sekvencom.

\begin{table}[]
\centering
\caption{Usporedba vlastitog rješenja (SpaCRO) s sekvencom iz faze razmještanja (default) i referentnim radom (Sparc).}
\label{tbl:usporedba}
\begin{tabular}{l|ll}
  & \multicolumn{1}{l}{lambda} & ecoli \\ \hline
  default & 86.16                       & 88.57 \\
  Sparc   & \textbf{95.41}              & \textbf{98.17} \\
  SpaCRO  & 90.97                       & 95.69
\end{tabular}
\end{table}

Možemo uočiti kako je postignuto značajno poboljšanje sekvence genoma, ali rezultati dobiveni referentnim algoritmom su ipak bolji.
Razlog nepostizanja visokih rezultata može biti zbog malih razlika u implementaciji algoritma.
Također, originalni \emph{Sparc} provodi algoritam u nekoliko iteracija.
Na takav način svakom iteracijom malo raste točnost generirane sekvence (zasićenje je već nakon 3-5 iteracija).
Mi nismo uočili poboljšanje performansi s povećanjem iteracija.

\section{Memorija i vrijeme izvođenja}

Kao dio projekta, potrebno je izmjeriti potrošnju memorije i vrijeme izvođenja našega algoritma te zadovoljiti određena ograničenja.
Mjerenje je obavljeno pomoću alata cgmemtime\footnote{github.com/isovic/cgmemtime}, a rezultati su prikazani u tablici \ref{tbl:mem}.
I utrošena memorija i vrijeme izvođenja zadovoljavaju zadana ograničenja.

\begin{table}[]
\centering
\caption{Potrošnja memorije i vrijeme izvođenja našeg algoritma na testnim skupovima podataka.}
\label{tbl:mem}
\begin{tabular}{l|ll}
  & lambda & ecoli \\ \hline
  mem {[}MB{]} & 30     & 2464  \\
  time {[}s{]} & 0.45   & 62.8 
\end{tabular}
\end{table}

